\begin{itshape}
    \textbf{Abstract:} \\

Automated Driving Systems (ADS) critically depend on robust perception systems capable of generating reliable environmental representations. Bird's-Eye View (BEV) maps are fundamental for this, as they offer an unified, top-down 2D representation of the vehicle's immediate surroundings. While traditional methods for generating BEV semantic segmentation maps from camera data rely on a previous semantic segmentation of the perspective camera-image followed by Inverse Perspective Mapping (IPM), this approach is limited by assumptions of flat ground and distortion for taller objects. While recent data-driven techniques address these limitations, the specific impact of directly training semantic segmentation models on BEV images to see if it better segments planar elements remains underexplored.

This master's thesis investigates whether training a semantic segmentation model directly on BEV images outperforms the traditional image-space segmentation followed by IPM reprojection, particularly for planar elements as they get a metric representation in the BEV domain. To address this, a novel BEV2Seg\_2 framework is developed, enabling a direct comparison between these two strategies. Contrary to the initial hypothesis, experiments consistently demonstrated that the traditional Segmenting-Then-IPM strategy yielded superior segmentation performance for both obstacles and planar elements. Additionally, this work found that modifying camera extrinsic parameters for BEV data augmentation was more effective than traditional geometric augmentations, significantly reducing model overfitting.

Beyond this comparative analysis, this work also explores a practical application by developing an automated pre-annotation pipeline to generate BEV masks for occupancy, occlusion, and driveable areas.  While this pipeline has some limitations, especially with partially occluded objects, it produces reliable enough masks for valuable downstream tasks in autonomous driving, such as scene reconstruction and path planning. 

In summary, this thesis provides a robust experimental foundation demonstrating that traditional Segmenting-Then-IPM offers superior performance in BEV semantic segmentation, and presents a functional pre-annotation system that advances autonomous driving perception.

    \textbf{Keywords: Automated Driving Systems (ADS), BEV Semantic Segmentation, Automated Annotation, Computer Vision, Occupancy Mapping, Occlusion Estimation}
\end{itshape}
\newpage
