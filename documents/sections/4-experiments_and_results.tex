% ================================================
% =                 EXPERIMENTS                  = 
% ================================================ 

This section includes all experiments carried out for evaluating the difference between the two approaches considered in the BEV2Seg\_2 pipeline, experiments to study what is the influence of extrinsic parameters modification as data augmentation technique for semantic segmentation of BEV images and the final evaluation of the proposed annotation pipeline of occupancy, occlusion and driveable areas.

\subsection{BEV2Seg\_2}
Both approaches, segmenting-then-IPM and IPM-then-segmenting, were trained under the same conditions: the encoder layers were not frozen during fine-tuning for the segmentation task, allowing the entire model to adapt to the data. Input image preprocessing was performed using the \textit{SegformerImageProcessor}, which includes resizing the images to $512 \times 512$ pixels, rescaling pixel values by a factor of $1/255$, and normalizing using the mean and standard deviation values from ImageNet~\cite{imagenet}. This ensures the input format is consistent with what the pretrained encoders expect.

Additionally, during training, semantic masks were provided with the \texttt{reduce\_labels} parameter set to \texttt{False}, since our dataset includes the "background" class. This means gradients are computed for all image regions during training. All experiments were carried out using the same hardware configuration described in Table~\ref{tab:hardware}. Among the six available Segformer encoder variants, three were selected for evaluation: MiT-b0, MiT-b2, and MiT-b4.

\hl{Complete this table.}
\begin{table}[h]
    \centering
    \begin{tabular}{c l c}
        \toprule
        \textbf{Component} & \textbf{Specifications} & \textbf{Num workers} \\
        \midrule
        CPU         & - & 8 \\
        GPU         & - & 2 \\      
        Memmory     & - & - \\
        OS          & - & - \\
        \bottomrule
    \end{tabular}
    \caption{ Hardware used for experiments }
    \label{tab:hardware}
\end{table}

Regarding to the used notation, the models that follow the segment-then-IPM pipeline to obtain BEV semantic masks are referred to as \texttt{raw2seg\_bev}, while those that first apply IPM and then perform segmentation are named \texttt{raw2bev\_seg}.

The two approaches were firstly trained using the smallest Segformer model variant, MiT-b0, for $20.7K$ steps without applying any regularization techniques. This initial experiment was carried out to observe whether the models were able to learn and predict on the dataset and see if them suffered from overfitting. Also, for this purpose, the choice of MiT-b0 was intentional as it trains faster and, due to its limited capacity, is less prone to extreme overfitting compared to larger models. This made it a suitable candidate for testing different hyperparameter configurations in a lightweight environment.

\hl{I didn't find what is the segformer's loss function. Maybe Categorical Cross-Entropy Loss.}
\hl{Learning rate scheduler: polynomial.}
\hl{Comment that with model size MiT-b4 gradient accumulation techniques were used to fit the GPU memory specs.}
\hl{Comment the max epochs set to 200 and the checkpoint strategy used: it is controvesial wheter to use the loss or mIoU. Whe have decided to use the loss.}
\hl{Comment the training stopping strategy is based on the eval loss.}

\begin{figure}[h!]
    \centering
    \includegraphics[width=0.7\linewidth]{./images/experiments/overfitting_bev_nu.png}
    \caption{Training and evaluation loss of raw2seg\_bev and raw2bev\_seg MiT-b0 models without any regularization technique}
    \label{fig:overfitting_mit-b0}
\end{figure}

As shown in Figure~\ref{fig:overfitting_mit-b0}, both models showed clear signs of overfitting. While the training loss continued to decrease continuously, the validation loss began to increase, indicating a lack of generalization and failure to converge. These results highlight the importance of introducing regularization techniques even for small model sizes. Additionally, no signs of exploding gradients were observed during the training of these models.

Two main approaches were selected to tackle the overfitting problem: weight decay (also known as L2 regularization) and data augmentation. Weight decay penalizes large weights during training, and makes the model more robust and less prone to memorizing irrelevant details; while data augmentation techniques introduces variability in the training dataset enabling the model to adapt better to unseen data. However, the introduction of data augmentation techniques into \aclink{BEV} images domain is not trivial which raises another research question.

With all of this in place, the training strategy is designed to support experimentation in order to answer two main research questions:

\begin{itemize}
    \item \textit{Which data augmentation technique is more effective: using the same strategy applied in the other approach, or modifying the extrinsic camera parameters?}
    \item \textit{Which of the two approaches performs better for BEV image segmentation?}
\end{itemize}

\subsubsection{BEV images data augmentation techniques comparison}
Data augmentations are commonly used in deep learning models to mitigate overfitting during training and improve model generalization. There exists multiple types of data augmentation on the image domain: from pixel-based transformations, such as color space modifications, histogram equalization or filtering operations; to geometric transformations, including translations, rotations, shearings and homographies. These techniques have been widely applied in computer vision tasks and have shown to enhance model performance. However, performing data augmentation in \aclink{BEV} is not an easy task, as \aclink{IPM} images are already homographies of camera images, resulting in inherent distorsions.

Filtering operations can be applied to both standard and \aclink{BEV} images but geometric transformations were selected as the primary data augmentation method for camera domain images following the strategies employed in training the SegFormer model \cite{segformer}. Accordingly, random resizing, random cropping, and horizontal flipping were chosen as augmentation operations for perspective images.

Regarding \aclink{BEV} data augmentations, some multi-view methods implement strategies such as random flipping and random scaling, while others operate in the frequency domain \cite{HSDA}. However, these approaches apply augmentations to perspective images before the BEV transformation. Performing random cropping on a \aclink{BEV} image may lead to significant information loss, as large portions of the image may consist of unlabeled background data, potentially resulting in crops with insufficient information for effective training (Figure~\ref{fig:bev_cropping}). 

\begin{figure}[h]
    \centering
    % Row labels
    \setlength{\tabcolsep}{1pt}  % Reduce column padding
    \renewcommand{\arraystretch}{0.5}
    \begin{tabular}{c c c c c c c c}

        \includegraphics[width=0.12\textwidth]{images/methodology/mini/mini_0_bev.png} & 
        \includegraphics[width=0.12\textwidth]{images/methodology/mini/mini_1_bev.png} & 
        \includegraphics[width=0.12\textwidth]{images/methodology/mini/mini_2_bev.png} &
        \includegraphics[width=0.12\textwidth]{images/methodology/mini/mini_3_bev.png} & 
        \includegraphics[width=0.12\textwidth]{images/methodology/mini/mini_4_bev.png} & 
        \includegraphics[width=0.12\textwidth]{images/methodology/mini/mini_5_bev.png} &
        \includegraphics[width=0.12\textwidth]{images/methodology/mini/mini_6_bev.png} &
        \includegraphics[width=0.12\textwidth]{images/methodology/mini/mini_7_bev.png} \\ 
        
        \includegraphics[width=0.12\textwidth]{images/methodology/data_augmentations/bev_crop_image_0_9.png} & 
        \includegraphics[width=0.12\textwidth]{images/methodology/data_augmentations/bev_crop_image_1_2.png} & 
        \includegraphics[width=0.12\textwidth]{images/methodology/data_augmentations/bev_crop_image_2_5.png} &
        \includegraphics[width=0.12\textwidth]{images/methodology/data_augmentations/bev_crop_image_3_5.png} & 
        \includegraphics[width=0.12\textwidth]{images/methodology/data_augmentations/bev_crop_image_4_3.png} & 
        \includegraphics[width=0.12\textwidth]{images/methodology/data_augmentations/bev_crop_image_5_2.png} &
        \includegraphics[width=0.12\textwidth]{images/methodology/data_augmentations/bev_crop_image_6_1.png} &
        \includegraphics[width=0.12\textwidth]{images/methodology/data_augmentations/bev_crop_image_7_2.png} \\ 

        \includegraphics[width=0.12\textwidth]{images/methodology/data_augmentations/bev_crop_mask_0_9.png} & 
        \includegraphics[width=0.12\textwidth]{images/methodology/data_augmentations/bev_crop_mask_1_2.png} & 
        \includegraphics[width=0.12\textwidth]{images/methodology/data_augmentations/bev_crop_mask_2_5.png} &
        \includegraphics[width=0.12\textwidth]{images/methodology/data_augmentations/bev_crop_mask_3_5.png} & 
        \includegraphics[width=0.12\textwidth]{images/methodology/data_augmentations/bev_crop_mask_4_3.png} & 
        \includegraphics[width=0.12\textwidth]{images/methodology/data_augmentations/bev_crop_mask_5_2.png} &
        \includegraphics[width=0.12\textwidth]{images/methodology/data_augmentations/bev_crop_mask_6_1.png} &
        \includegraphics[width=0.12\textwidth]{images/methodology/data_augmentations/bev_crop_mask_7_2.png} \\

    \end{tabular}
    
    \caption{Random cropping and horizontal flipping on BEV images. Original BEV images on first row; random flipped and cropped images on second row and corresponding semantic masks on last row.}
    \label{fig:bev_cropping}
\end{figure}

In this context, a different approach was also considered: applying geometric transformations by modifying the camera's extrinsic parameters before reprojecting to \aclink{BEV} space. The objective is to introduce random transformations along one of the camera’s rotation axes, generating diverse \aclink{BEV} reprojections with varying degrees of distortion. This technique may enable the model to adapt to different extrinsic camera configurations, improving its robustness to variations in camera placement and orientation (Figure~\ref{fig:bev_data_aug}).

\begin{figure}[h]
    \centering
    % Row labels
    \setlength{\tabcolsep}{1pt}  % Reduce column padding
    \renewcommand{\arraystretch}{0.5}
    \begin{tabular}{c c c c c c}
        & $-0.25$ rad & $-0.125$ rad & $0.0$ rad & $0.125$ rad & $0.25$ rad \\ 
        
        \rotatebox{90}{\textbf{Yaw}} & 
        \includegraphics[width=0.15\textwidth]{images/methodology/data_augmentations/rx_-0.25_0.png} & 
        \includegraphics[width=0.15\textwidth]{images/methodology/data_augmentations/rx_-0.125_1.png} & 
        \includegraphics[width=0.15\textwidth]{images/methodology/data_augmentations/rx_0.0_2.png} & 
        \includegraphics[width=0.15\textwidth]{images/methodology/data_augmentations/rx_0.125_3.png} & 
        \includegraphics[width=0.15\textwidth]{images/methodology/data_augmentations/rx_0.25_4.png} \\ 
        
        \rotatebox{90}{\textbf{Pitch}} & 
        \includegraphics[width=0.15\textwidth]{images/methodology/data_augmentations/rz_-0.25_0.png} & 
        \includegraphics[width=0.15\textwidth]{images/methodology/data_augmentations/rz_-0.125_1.png} & 
        \includegraphics[width=0.15\textwidth]{images/methodology/data_augmentations/rz_0.0_2.png} & 
        \includegraphics[width=0.15\textwidth]{images/methodology/data_augmentations/rz_0.125_3.png} & 
        \includegraphics[width=0.15\textwidth]{images/methodology/data_augmentations/rz_0.25_4.png} \\ 

        \rotatebox{90}{\textbf{Roll}} & 
        \includegraphics[width=0.15\textwidth]{images/methodology/data_augmentations/ry_-0.25_0.png} & 
        \includegraphics[width=0.15\textwidth]{images/methodology/data_augmentations/ry_-0.125_1.png} & 
        \includegraphics[width=0.15\textwidth]{images/methodology/data_augmentations/ry_0.0_2.png} & 
        \includegraphics[width=0.15\textwidth]{images/methodology/data_augmentations/ry_0.125_3.png} & 
        \includegraphics[width=0.15\textwidth]{images/methodology/data_augmentations/ry_0.25_4.png} \\ 
        
    \end{tabular}
    
    \caption{Effect of camera transformations on BEV projection. The first row shows variations in the yaw axes, the second in pitch, and the third in roll.}
    \label{fig:bev_data_aug}
\end{figure}



\hl{Review this!!!}
Random cropping, flipping and rescaling augmentation strategies were applied to both normal and \aclink{BEV} images for introducing variability in the training data. With these regularization techniques in place, the overfitting behavior was significantly reduced as shown in Figure~\ref{fig:before_after_data_aug}.


\hl{Temporal graph!!}
\begin{figure}[h!]
    \centering
    \includegraphics[width=0.7\linewidth]{./images/experiments/before_an_after_data_aug.png}
    \caption{Before and after regularization techniques. Light colors before, darker colors after. BEV segmentation model uses normal data augmentations.}
    \label{fig:before_after_data_aug}
\end{figure}

It is important to note that the evaluation datasets differ between the two pipelines as one evaluates performance on BEV projections while the other does so on regular images motivating the gap in their validation results.

\hl{This raises an important question: \textit{Which approach is more effective: traditional geometric data augmentation techniques or the modification of extrinsic parameters?}}

\hl{Comparison between normal geometric data augmentation techniques and camera's extrinsic parameters modifications.}

The objective of this section is to show what of the data augmentation techniques is better for \aclink{BEV} semantic segmentation training and how the modification of the camera extrinsic's parameters affects the semantic predictions.

Despite significantly reducing the overfitting, normal data augmentations for \aclink{BEV} images


\subsubsection{Pipeline comparison}
\hl{raw2seg\_bev results compared with raw2bev\_seg results.}



\subsection{3D detections evaluations}
\hl{Explain the selected NuScenes selected vehicle scene, and expose the metrics used (mIoU and v2v distance).}


\begin{figure}[h]
    \centering
    % Row labels
    \setlength{\tabcolsep}{1pt}  % Reduce column padding
    \renewcommand{\arraystretch}{0.5}
    \begin{tabular}{c c}
        
        \includegraphics[width=0.45\textwidth]{images/experiments/frame_0_3d_scene.png} &
        \includegraphics[width=0.45\textwidth]{images/experiments/frame_2_3d_scene.png} \\
        
        \includegraphics[width=0.45\textwidth]{images/experiments/frame_0_3d_bb.png} &
        \includegraphics[width=0.45\textwidth]{images/experiments/frame_2_3d_bb.png} \\

    \end{tabular}
    
    \caption{Bounding Box Disparity scene calculation. (a) shows an example of intersection, while (b) shows an example of volumetric distance}
    \label{fig:bbox_disparity}
\end{figure}


\subsection{BEV masks evaluation}
\hl{Groundtruth BEV masks could be generated from annotations and compute mIoU between the annotated BEV masks and ground truth ones.}

