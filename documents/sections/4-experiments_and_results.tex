
This section includes all experiments carried out for evaluating the difference between the two approaches considered in the BEV2Seg\_2 pipeline, experiments to study what is the influence of extrinsic parameters modification as data augmentation technique for semantic segmentation of BEV images and the final evaluation of the proposed annotation pipeline of occupancy, occlusion and driveable areas.

\subsection{BEV2Seg\_2}
No se han congelado las capas del encoder para realizar el fine-tuning en el task de segmentación. El preprocesado de las imágenes se ha realizado con \textit{SegformerImageProcessor} aplicando un cambio de tamaño de la imagen a $512 \times 512$, un re-escalado con un factor de $1/255$y un normalizado con los valores de media y desviación típica de ImageNet. De esta forma, aseguramos que el preprocesado realizado es el mismo que el esperado por el encoder.

Además, a la hora de proporcionar las máscaras semánticas de entrenamiento, el parámetro "reduce\_labels" se ha mantenido en \textit{False}, ya que no contamos con la clase "background". Aquellas zonas que no se quieren inferir, como son el fondo y las áreas no etiquetadas, tienen el label "ignore" así que no se calcula gradiente para ellas.
